Minha ideia com esse material é documentar uma trajetória de aprendizado em métodos numéricos de diferenças finitas com uma perspectiva computacional, vamos implementar as Equações Diferenciais mais conhecidas, utilizando a linguagem \textit{Python}, além disso, todos os códigos aqui desenvolvidos serão disponibilizados no \textit{GitHub}, para consultas e projetos futuros. 

Vamos partir do pressuposto que você, leitor, é um estudante que tem familiaridade com a linguagem Python, e nesse sentido não vemos a necessidade de fazer uma introdução a linguagem. A matemática necessária para acompanhar o que será escrito aqui não será muito avançada, visto que nosso foco não é demonstrar teoremas e nem deduzir métodos numéricos muito complicados. 

Vamos iniciar nosso primeiro capítulo deduzindo os principais métodos em diferenças finitas, Diferença progressiva, regressiva e centrada, para primeira e segunda ordem, não se engane, somente esses métodos podem dar muito trabalho a depender do objetivo que queremos. No segundo capítulo começaremos descrevendo e entendendo a equação de Poisson, e quais fenômenos são modelados por ela, depois disso, vamos para a implementação e resolução da equação unidimensional e bidimensional. 

No Terceiro capítulo abordaremos a equação do calor com a metodologia utilizada para a equação de Poisson, então, vamos descrevê-la, entendendo os fenômenos que ela pode modelar. Essas duas equação são diferentes, e vamos discutir essa diferença ao passo que resolvemos problemas com elas.  